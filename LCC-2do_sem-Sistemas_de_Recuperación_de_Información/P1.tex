\documentclass[11pt]{article}

\usepackage{graphicx}
\usepackage[top=0.25in, left=1.0in, right=1.0in]{geometry}
\setlength{\topmargin}{-1.0in}
\usepackage{url}
\usepackage{enumitem}
\usepackage{tabularx}
\usepackage{setspace}
\usepackage{tikz}
\usetikzlibrary{er,positioning}  
\usetikzlibrary{decorations.pathreplacing}
\usetikzlibrary{arrows, decorations.markings}
\usetikzlibrary{shapes.geometric}
\usepackage{amsmath}

\usepackage[spanish]{babel}
\usepackage[utf8]{inputenc}

\usepackage{datetime}
\newdate{date}{16}{12}{2023}
\date{\displaydate{date}}

\usepackage{listingsutf8}
\usepackage{xcolor,listings}
\usepackage{textcomp}
\usepackage{color}

\usepackage{dirtree}

\definecolor{codegreen}{rgb}{0,0.6,0}
\definecolor{codegray}{rgb}{0.5,0.5,0.5}
\definecolor{codepurple}{HTML}{C42043}
\definecolor{backcolour}{HTML}{F2F2F2}
\definecolor{red}{HTML}{FF0000}
\definecolor{bookColor}{cmyk}{0,0,0,0.90}  
\color{bookColor}

\usepackage{multirow}
\usepackage{array}
\usepackage{longtable}

\lstset{upquote=true}

\lstdefinestyle{mystyle}{
	% backgroundcolor=\color{backcolour},   
	commentstyle=\color{codegreen},
	keywordstyle=\color{codepurple},
	numberstyle=\numberstyle,
	stringstyle=\color{codepurple},
	basicstyle=\footnotesize\ttfamily,
	breakatwhitespace=false,
	breaklines=true,
	captionpos=b,
	keepspaces=true,
	% numbers=left,
	numbersep=10pt,
	showspaces=false,
	showstringspaces=false,
	showtabs=false,
	otherkeywords = {OUT, SIGNAL, DECLARE, BEFORE, before, AFTER, CALL, WHILE, PROCEDURE, if, if},
}
\lstset{style=mystyle}

\newcommand\numberstyle[1]{%
	\footnotesize
	\color{codegray}%
	\ttfamily
	\ifnum#1<10 0\fi#1 |%
}

\newcommand{\linetomark}{\rule{0.5cm}{0.4pt} }

\begin{document}
	
	\lstset{
		literate=%
		{á}{{\'a}}1
		{í}{{\'i}}1
		{é}{{\'e}}1
		{ý}{{\'y}}1
		{ú}{{\'u}}1
		{ó}{{\'o}}1
		{ě}{{\v{e}}}1
		{š}{{\v{s}}}1
		{č}{{\v{c}}}1
		{ř}{{\v{r}}}1
		{ž}{{\v{z}}}1
		{ď}{{\v{d}}}1
		{ť}{{\v{t}}}1
		{ň}{{\v{n}}}1                
		{ů}{{\r{u}}}1
		{Á}{{\'A}}1
		{Í}{{\'I}}1
		{É}{{\'E}}1
		{Ý}{{\'Y}}1
		{Ú}{{\'U}}1
		{Ó}{{\'O}}1
		{Ě}{{\v{E}}}1
		{Š}{{\v{S}}}1
		{Č}{{\v{C}}}1
		{Ř}{{\v{R}}}1
		{Ž}{{\v{Z}}}1
		{Ď}{{\v{D}}}1
		{Ť}{{\v{T}}}1
		{Ň}{{\v{N}}}1                
		{Ů}{{\r{U}}}1    
	}
	
	\begin{centering}
		\huge Modelo de Planificación y\\Control del Proceso Docente para\\Sistemas de Recuperación de Información
		
	\end{centering}
	
	\vspace{2\baselineskip}
	
	\textbf{Año:} 3ro 
	
	\textbf{Semestre:} 2do 
	
	\textbf{Curso:} 2024 
	
	\textbf{Carrera:} Ciencia de la Computación 
		
	\textbf{Plan:} E 
	
	
	\subsection*{Planificación de las clases}
	
	Debido a la reducción de las horas/clases del curso 2024, para este semestre se propone la siguiente descomposición de temas por clase. 
	
	\begin{longtable}{|>{\centering\arraybackslash}p{1.5cm}|p{3.7cm}|p{2.2cm}|p{6cm}|}
		\hline
		\multicolumn{1}{|c|}{\textbf{Semana}} & \multicolumn{1}{c|}{\textbf{Día}} & \multicolumn{1}{c|}{\textbf{Modalidad}} & \multicolumn{1}{c|}{\textbf{Contenido}} \\
		\hline 
		\endfirsthead
		%\hline
		%\multicolumn{4}{|l|}{{Continuación de la página anterior}} \\
		\hline
		\multicolumn{1}{|c|}{\textbf{Semana}} & \multicolumn{1}{c|}{\textbf{Día}} & \multicolumn{1}{c|}{\textbf{Modalidad}} & \multicolumn{1}{c|}{\textbf{Contenido}} \\
		\hline 
		\endhead
		\hline
		\endfoot
		\hline
		\endlastfoot
		
		1 & lunes, 29 de abril  & Conferencia & \begin{minipage}[t]{\linewidth}
			\begin{itemize}
				\item Introducción a los SRI
				\item Definición de un SRI \\
			\end{itemize}
		\end{minipage}  \\
		\cline{2-4}
		& miércoles, 1 de mayo & \multicolumn{2}{c|}{------ Feriado ------} \\
		\hline
		\hline
		
		2 & lunes, 6 de mayo  & Conferencia & \begin{minipage}[t]{\linewidth}
			\begin{itemize}
				\item Presentación de los Modelos Booleano, Vectorial y Probabilístico 
				\item Ventajas y desventajas de cada uno \\
			\end{itemize}
		\end{minipage} \\
		\cline{2-4}
		& miércoles, 8 de mayo  & Laboratorio & Implementación del Modelo Booleano \\
		\hline
		\hline
		
		3 & lunes, 13 de mayo  & Conferencia & \begin{minipage}[t]{\linewidth}
			\begin{itemize}
				\item Evaluación y retroalimentación 
				\item Expansión de la consulta \\
			\end{itemize}
		\end{minipage} \\
		\cline{2-4}
		& miércoles, 15 de mayo  & Laboratorio & Implementación del Modelo Vectorial \\
		\hline
		\hline
		
		4 & lunes, 20 de mayo  & Conferencia & \begin{minipage}[t]{\linewidth}
			\begin{itemize}
				\item Representación del conocimiento
				\item Indexación y Almacenamiento \\
			\end{itemize}
		\end{minipage}  \\
		\cline{2-4}
		& miércoles, 22 de mayo  & Laboratorio & Implementación del Modelo Vectorial (Continuación) \\
		\hline
		\hline
		
		5 & lunes, 27 de mayo  & Conferencia & \begin{minipage}[t]{\linewidth}
			\begin{itemize}
				\item Procesamiento del lenguaje natural proveniente de textos \\
			\end{itemize}
		\end{minipage} \\
		\cline{2-4}
		& miércoles, 29 de mayo  & Laboratorio & Expansión de consultas \\
		\hline
		\hline
		
		6 & lunes, 3 de junio  & Conferencia & \begin{minipage}[t]{\linewidth}
			\begin{itemize}
				\item Procesamiento de sonidos \\
			\end{itemize}
		\end{minipage} \\
		\cline{2-4}
		& miércoles, 5 de junio  & Laboratorio & Evaluación de los SRI \\
		\hline
		\hline
		
		7 & lunes, 10 de junio  & Conferencia & \begin{minipage}[t]{\linewidth}
			\begin{itemize}
				\item Procesamiento de imágenes \\
			\end{itemize}
		\end{minipage}  \\
		\cline{2-4}
		& miércoles, 12 de junio  & Laboratorio &  Procesamiento del lenguaje natural\\
		\hline
		\hline
		
		8 & lunes, 17 de junio  & Conferencia & \begin{minipage}[t]{\linewidth}
			\begin{itemize}
				\item Eras de la Web
				\item Minería en la Web: crawling y scraping \\
			\end{itemize}
		\end{minipage} \\
		\cline{2-4}
		& miércoles, 19 de junio  & Laboratorio & Procesamiento de imágenes \\
		\hline
		\hline
		
		9 & lunes, 24 de junio  & Conferencia & \begin{minipage}[t]{\linewidth}
			\begin{itemize}
				\item Ranking en la Web
				\item SEO (posicionamiento) \\
			\end{itemize}
		\end{minipage} \\
		\cline{2-4}
		& miércoles, 26 de junio  & Laboratorio & Procesamiento de sonidos \\
		\hline
		\hline
		
		10 & lunes, 1 de julio  & Conferencia & \begin{minipage}[t]{\linewidth}
			\begin{itemize}
				\item Minería en redes
				\item Métodos para la extracción de información, con vista de la recuperación \\
			\end{itemize}
		\end{minipage}  \\
		\cline{2-4}
		& miércoles, 3 de julio  & Laboratorio & Web Crawling y Scraping\\
		\hline
		\hline
		
		11 & lunes, 8 de julio  & Conferencia & \begin{minipage}[t]{\linewidth}
			\begin{itemize}
				\item Recuperación de información con conjuntos masivos de datos
				\item Métodos para la extracción de información, con vista de la recuperación 
				\item Hadoop 
				\item Optimización en el espacio de las búsquedas \\
			\end{itemize}
		\end{minipage}  \\
		\cline{2-4}
		& miércoles, 10 de julio  & Laboratorio & Minería de redes  \\
		\hline
		\hline
		
		12 & lunes, 15 de julio  & Conferencia & Tiempo extra  \\
		\cline{2-4}
		& miércoles, 17 de julio  & Laboratorio & Trabajo de Control Parcial (referente a la web) \\
		\hline
	\end{longtable}
	
	
	\vspace{2mm}
	
	\subsection*{Sistema evaluativo}
	
	Se propone el siguiente sistema de calificación:
	
	\begin{enumerate}
		\item Proyecto Investigativo \\[1mm]
		El objetivo es la investigación por parte del estudiante sobre problemas actuales referentes o vinculados a las temáticas a tratar durante el curso. Esta investigación tiene que culminar en una implementación donde el estudiante proponga una solución al problema seleccionado a trabajar. Los temas están vinculado a la recomendación de información.
		
		\item Trabajo de Control Parcial \\[1mm]
		El objetivo es evaluar la parte del curso referente a la Web.
		
	\end{enumerate}
	
	
	
	
\end{document}


